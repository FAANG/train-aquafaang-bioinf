\documentclass[12pt]{article}
\usepackage[dvipsnames]{xcolor}
\usepackage{times}
\usepackage[T1]{fontenc}
\usepackage{float}
\usepackage[margin=1.2in]{geometry}
\usepackage{graphicx}
\usepackage{textcomp}
\usepackage{titlesec}
\usepackage{enumitem}
\usepackage{adjustbox}

\graphicspath{ {./images/} }

\usepackage[colorlinks = true,
linkcolor = blue,
urlcolor  = blue,
citecolor = blue,
anchorcolor = blue]{hyperref}
\urlstyle{same}
\newcommand{\MYhref}[3][blue]{\href{#2}{\color{#1}{#3}}}%

% paragraphs are also numbered
\setcounter{secnumdepth}{3}
\titleformat{\paragraph}
{\normalfont\normalsize\bfseries}{\theparagraph}{1em}{}
\titlespacing*{\paragraph}
{0pt}{3.25ex plus 1ex minus .2ex}{1.5ex plus .2ex}

%No indent
\setlength{\parindent}{0pt}
\setenumerate{itemsep=0pt,parsep=0pt}


% https://en.wikibooks.org/wiki/LaTeX/Source_Code_Listings
\usepackage{listings, lstautogobble, beramono, tikz}

% Copy and paste text from boxes
\lstset{ 
	upquote=true,
	columns=fullflexible,
	basicstyle=\ttfamily,
	literate={*}{{\char42}}1
	{-}{{\char45}}1
	{\ }{{\copyablespace}}1
}
\usepackage[space=true]{accsupp}
\newcommand{\copyablespace}{\BeginAccSupp{method=hex,unicode,ActualText=00A0}\hphantom{x}\EndAccSupp{}}

% Make boxes for code
\lstset{
	basicstyle=\small\ttfamily,
	columns=flexible,
	breaklines=true,
	breakatwhitespace=false,
	backgroundcolor=\color{blue!20},
	showstringspaces=false,
	frameround=ffff,
	frame=single,
	rulecolor=\color{black},
	autogobble=true
}
\definecolor{mypink1}{rgb}{0.858, 0.188, 0.478}
% Highlight keywords
\lstset{
	emph=[1]{%  
		wget, gunzip, cd, docker,  grep, cut, sort, export%
	},emphstyle=[1]{\color{red}\bfseries},%
	emph=[3]{%  
		--mount, --workdir, -e%
	},emphstyle=[3]{\color{purple}\bfseries},%
	emph=[4]{%
		Salmo\_salar.ICSASG\_v2.dna.toplevel.fa.gz, Salmo\_salar.ICSASG\_v2.dna.toplevel.fa,  Salmo\_salar.ICSASG\_v2.dna.toplevel.fa.fai,
		Ssal\_chrm.txt
	},emphstyle=[4]{\color{blue}\bfseries},%
	emph=[5]{%  
		juettemann\/samtools:latest,  juettemann/genrich:1.0%
	},emphstyle=[5]{\color{brown}\bfseries},%
	emph=[6]{%  
		run, inspect, exec
	},emphstyle=[6]{\color{mypink1}\bfseries},%
	emph=[7]{%  
		run, inspect, exec
	},emphstyle=[7]{\color{red}\bfseries},%
	emph=[8]{%  
		\$blast\_c
	},emphstyle=[8]{\color{TealBlue}\itshape},%
}%
%\lstdefinelanguage{mylang}{ %
%	alsoletter=.\_/:,%
%}%

%\makeatletter
%\newenvironment{btHighlight}[1][]
%{\begingroup\tikzset{bt@Highlight@par/.style={#1}}\begin{lrbox}{\@tempboxa}}
%	{\end{lrbox}\bt@HL@box[bt@Highlight@par]{\@tempboxa}\endgroup}
%
%\newcommand\btHL[1][]{%
%	\begin{btHighlight}[#1]\bgroup\aftergroup\bt@HL@endenv%
%	}
%	\def\bt@HL@endenv{%
%	\end{btHighlight}%   
%	\egroup
%}
%\newcommand{\bt@HL@box}[2][]{%
%	\tikz[#1]{%
%		\pgfpathrectangle{\pgfpoint{1pt}{0pt}}{\pgfpoint{\wd #2}{\ht #2}}%
%		\pgfusepath{use as bounding box}%
%		\node[anchor=base west, fill=orange!30,outer sep=0pt,inner xsep=1pt, inner ysep=0pt, rounded corners=3pt, minimum height=\ht\strutbox+1pt,#1]{\raisebox{1pt}{\strut}\strut\usebox{#2}};
%	}%
%}
%\makeatother


\usepackage{fancyhdr}
\renewcommand{\headrulewidth}{.2mm} % header line width

\pagestyle{fancy}
\fancyhf{}
%\fancyhfoffset[L]{2cm} % left extra length
%\fancyhfoffset[R]{2cm} % right extra length
\rhead{\today}
\lhead{\bfseries Container tutorial}
\rfoot{}
\rhead{\includegraphics[width=1cm]{logo}}


\begin{document}
	
	% Article top matter
	\title{Container} 
	\author{Thomas Juettemann, EMBL-EBI\\
	\texttt{aqua\_faang\_course@ebi.ac.uk}}  %\texttt formats the text to a typewriter style font
	\date{May 10-12, 2021}  %\today is replaced with the current date
	\maketitle
	
	
	\section{Introduction}
	
	\section{Docker Hub}
	Docker Hub is the world’s largest repository of container images, at the time of this course it hosts 8.3 million repositories. 
	In this section we explore a couple of repositories and run our first container.
	
		\subsection{Hello World}
			Because that is how every IT adventure begins.
			
			
			\paragraph{Task}
				\fcolorbox{black}[HTML]{E9F0E9}{\parbox{\linewidth}{%	
						\begin{enumerate}
							\item On Dockerhub find the official  \textbf{hello-world} image and click on it.
							\item In the description page, find and execute the docker run command in the terminal.
						\end{enumerate}
				}}
			
			\paragraph{Solution}
				
				\begin{minipage}{\linewidth}
					\begin{lstlisting}
						https://hub.docker.com/_/hello-world
						docker run hello-world
					\end{lstlisting}
				\end{minipage}
	
	
		\subsection{Versions}
			Dockerhub offers to host several versions of a software.
			Here we take a quick look at the similarity and differences using PostgreSQL as an example.
			 
			\paragraph{Task}
				\fcolorbox{black}[HTML]{E9F0E9}{\parbox{\linewidth}{%	
						\begin{enumerate}
							\item On Dockerhub find the official  \textbf{postgres} image and click on it.
							\item Which is the latest version?
							\item What happens when you click on \textit{latest}?
							\item What is the difference between the first two tags in the first line?
							\item What is the major difference (hint: \textit{FROM}) between the first tag in line 1 and the first tag in line 2?
						\end{enumerate}
				}}
		
			\paragraph{Solution}
				\begin{minipage}{\linewidth}
					\begin{lstlisting}
						https://hub.docker.com/_/postgres
						# Latest version is 13.2
						# It is a link to the corresponding dockerfile which is hosted on GitHub 
						# They are different tags for the same dockerfile (=same image). Each line refers to the same image.
						# They use a different Linux version, debian:buster-slim vs alpine:3.13.  
					\end{lstlisting}
				\end{minipage}
	

		

	\section{BioContainers}
		BioContainers is a community-driven project that provides the infrastructure and basic guidelines to create, manage and distribute bioinformatics packages (e.g conda) and containers (e.g docker, singularity).
		Many tools used by computational biologists on a daily basis are readily available.
		A mayor difference to Docker Hub is that users can request container to be created. 
		
		
		\subsection{Exploring}
			In this section we will search for a specific tool on BioContainers, find the newest version, download the image and run it locally in different ways. 
			The tool of choice is BLAST, and we will use it to create a blast database and then run an alignment. 
			
			\subsubsection{Finding an image} 
				As a community driven project, BioContainers does not have the notion of an "official image", and therefore no such filter exists.
				Those images are usually used (=pulled) much more often than others.
				We exploit that fact to find the "official" NCBI version of BLAST.

				\paragraph{Task}
					\fcolorbox{black}[HTML]{E9F0E9}{\parbox{\linewidth}{%	
							\begin{enumerate}
								\item Got to \url{https://biocontainers.pro/}
								\item Click on Registry (wait for the search interface to load)
								\item Search for \textbf{blast}
								\item Below the search bar, change the sorting from "Default" to "Pull No" and the sort order from "Asc" to "Desc".
								\item The NCBI image is the first or second hit, click on it
							\end{enumerate}
					}}
				
				\paragraph{Solution}
					\begin{minipage}{\linewidth}
						\begin{lstlisting}
							# Final destination is:
							https://biocontainers.pro/tools/blast
						\end{lstlisting}
					\end{minipage}
		
			\subsubsection{Pulling a specfic version}
				The landing page contains generic information about the tool and different way of utilising it: Docker, Singularity and Conda.
				The version used in the various commands is not necessarily the most recent one.
				
					\paragraph{Task}
						\fcolorbox{black}[HTML]{E9F0E9}{\parbox{\linewidth}{%	
								\begin{enumerate}
									\item Click on "Packages and Containers"
									\item Find the version \textbf{quay.io/biocontainers/blast:2.11.0--pl526he19e7b1\_0}
									\item Copy the \textbf{docker pull} command
									\item Open a terminal
									\item Paste and execute the cached \textbf{docker pull} command
									\item Store (\textbf{=export}) the image name in an environment variable called \textit{blast\_c}
									\item List all images
								\end{enumerate}
						}}
					
					\paragraph{Solution}	
				
						\begin{minipage}{\linewidth}
							\begin{lstlisting}
								docker pull \
								quay.io/biocontainers/blast:2.11.0--pl526he19e7b1_0
								export blast_c=\
								'quay.io/biocontainers/blast:2.11.0--pl526he19e7b1_0'
								docker images #alternative: docker image ls
							\end{lstlisting}
						\end{minipage}
		
			\subsubsection{Inspecting an image}
				Time to play with the image!  and see if in the \textit{config section} \textbf{WorkingDir} or \textbf{Entrypoint} are set, and which \textbf{command} is executed when the container starts?
	
				\paragraph{Task}
					\fcolorbox{black}[HTML]{E9F0E9}{\parbox{\linewidth}{%	
						\begin{enumerate}
							\item \textit{Inspect} the image
							\item In the \textbf{Config} section, what are the values for 
							 \begin{enumerate}
							 	\item \textbf{WorkingDir}
							 	\item \textbf{Entrypoint}
							\end{enumerate}
								\item Which command (\textbf{Cmd}) is executed when the container is run?
						\end{enumerate}
					}}
					
				\paragraph{Solution}	
					
					\begin{minipage}{\linewidth}
						\begin{lstlisting}
							docker inspect $blast_c 
							docker inspect -f \
							'{{json .Config.WorkingDir}} \
							{{json .Config.Entrypoint}} \
							{{json .Config.Cmd}}' \ 
							$blast_c
							# "" null ["/bin/sh"]
						\end{lstlisting}
				\end{minipage}
		
			\subsubsection{From image to container}		
				A docker image is build from the instructions in a docker file.  			
				The image is then used as a template (or base), which we can copy and use to run an application. 
				The application needs an isolated environment in which to run: a container.
				From the BLAST image that we downloaded, we will create a container, and using that container we will create a BLAST database, and then query this database using blastp.
					
				The \textit{~/train-aquafaang-bioinf/container/data} directory contains a FASTA file with zebrafish proteins in FASTA format. 
				Using the blast container, we will create a BLAST database from the FASTA file.
				makeblastdb takes 2 arguments: 
				
				\begin{enumerate}
					\item -in path/to/input/fasta  
					\item -dbtype ()prot or nucl)
					\item Documentation: \url{https://ncbi.github.io/magicblast/cook/blastdb.html}
				\end{enumerate}
				
				Keep in mind that docker can not access any files in our local filesystem, unless it is mounted. 
				Remember that mount takes 3 arguments: 
				\begin{enumerate}
					\item type (usually bind)  
					\item source (absolute path to the directory outside the container)
					\item target (absolute path to the directory inside the container )
					\item Documentation:  \url{https://docs.docker.com/storage/}
				\end{enumerate}
			
		

			\paragraph{Task}
				\fcolorbox{black}[HTML]{E9F0E9}{\parbox{\linewidth}{%	
					\begin{enumerate}
						\item Using the BLAST container (=\$blast\_c), create a BLAST database with \textbf{makeblastdb} 						
					\end{enumerate}
				}}
			
			\paragraph{Solution}	
		
				\begin{minipage}{\linewidth}
					\begin{lstlisting}
					 docker run \
					 --mount \
					 type=bind, \
					 source=$HOME/train-aquafaang-bioinf/container/data, \
					 target=/mnt  \
					 $blast_c \
					 makeblastdb -in /mnt/zebrafish.1.protein.faa -dbtype prot
		 			\end{lstlisting}
				\end{minipage}
			
			\subsubsection{Life inside a container}
				To gain a better understanding of a container, we will run the actual alignment inside the container. 
				If you are not sure about the parameters, remember that docker offers help for each sub command (e.g. docker run --help).
				We want to know which user you are inside a container, and where all executables are located.
			
			
			
			\paragraph{Task}
				\fcolorbox{black}[HTML]{E9F0E9}{\parbox{\linewidth}{%	
						\begin{enumerate}
							\item \textbf{run} the BLAST container \textbf{interactive}, using a \textbf{pseudo tty} and bash as \textbf{command}. 
							\item Inside the container, run \textbf{whoami} and \textbf{which blastp} 
							\item List the contents of /mnt
						\end{enumerate}
				}}

			\paragraph{Solution}	
	
				\begin{minipage}{\linewidth}			
					\begin{lstlisting}
						docker run  -it  $blast_c bash
						# bash-4.2# whoami
						# root
						# bash-4.2# which blastp
						# /usr/local/bin/blastp
						# bash-4.2# ls -l /mnt
						# total 0
						exit
					\end{lstlisting}
				\end{minipage}	
			
				Now let us explore the effect of mounting.
			
				\paragraph{Task}
					\fcolorbox{black}[HTML]{E9F0E9}{\parbox{\linewidth}{%	
							\begin{enumerate}
								\item 	Run the same command as before, but also \textbf{mount} the \textbf{container/data} directory to \textbf{/mnt} 
							\end{enumerate}
					}}
	
				\paragraph{Solution}	
		
					\begin{minipage}{\linewidth}
						\begin{lstlisting}
						docker run -it 	--mount type=bind,source=$HOME/train-aquafaang-bioinf/container/data,target=/mnt  \
						$blast_c bash
					\end{lstlisting}
				\end{minipage}
			
 			\paragraph{Explore a running container}
		
	
			\paragraph{Task}
				\fcolorbox{black}[HTML]{E9F0E9}{\parbox{\linewidth}{%	
						\begin{enumerate}
							\item Open a second terminal
							\item List all running container
							\item Copy the container ID of the BLAST container
							\item Inspect it
							\item What is the \textbf{Cmd} in \textbf{Config}?
							\item  What do you see in the \textbf{Mounts} section?
						\end{enumerate}
				}}

			\paragraph{Solution}	
	
				\begin{minipage}{\linewidth}
					\begin{lstlisting}
						docker container ls
						docker inspect <container_id>
						OR 
						docker inspect -f '{{json .Config.Cmd}} {{json .Mounts}}' <container_id>
					\end{lstlisting}
				\end{minipage}		
	
			\paragraph{Execute a command in a running container}
				The docker exec command runs a new command in a running container.
			Using exec on the running container, execute a ls -l and a ls -l /mnt.
			What do you see?
	
			\paragraph{Task}
				\fcolorbox{black}[HTML]{E9F0E9}{\parbox{\linewidth}{%	
						\begin{enumerate}
							\item Using the container ID, execute a ls -l
							\item Using the container ID, execute a ls -l /mnt
						\end{enumerate}
				}}

			\paragraph{Solution}	
	
				\begin{minipage}{\linewidth}
					\begin{lstlisting}
						docker exec <container_id> ls -l /
						docker exec <container_id> ls -l /mnt
					\end{lstlisting}
				\end{minipage}
			
			\paragraph{Alignment}
			
			There are 3 ways of running the alignment:
			\begin{enumerate}
				\item Using the container the same way as we did with makeblastdb (swap makeblastdb for blastp). This is the most common solution in a workflow / pipeline or ad-hoc situation.
				\item Using docker exec while the first container is still running. This creates a basic version of a BLAST server that can be accessible to other users on the same server.
				\item Running blastp inside the first container. Mostly used for debugging or development.
			\end{enumerate}
			
			\paragraph{Task}
				\fcolorbox{black}[HTML]{E9F0E9}{\parbox{\linewidth}{%	
						\begin{enumerate}
							\item Run the alignment in all 3 different ways.
							\item Use different names for the out file
							\item To make the mounting short, create a environment variable:
							\textit{export docker\_mount="type=bind,source=\$HOME/train-aquafaang-bioinf/container/data,target=/mnt"}
							\item Exit the running container
						\end{enumerate}
				}}

			\paragraph{Solution}	
	
				\begin{minipage}{\linewidth}
					\begin{lstlisting}
						export docker_mount= \
						"type=bind,source=$HOME/train-aquafaang-bioinf/container/data,target=/mnt"
						# 1
						docker run \
						--mount $docker_mount \
						$blast\_c blastp \
						-query /mnt/P04156.fasta \
						-db /mnt/zebrafish.1.protein.faa \
						-out /mnt/result-1.txt
						# 2
						docker exec <container_id> \
						blastp -query /mnt/P04156.fasta -db /mnt/zebrafish.1.protein.faa \
						-out /mnt/result-2.txt
						# 3
						blastp -query /mnt/P04156.fasta -db /mnt/zebrafish.1.protein.faa \
						-out /mnt/result-3.txt
					\end{lstlisting}
				\end{minipage}
				
			
			\section{Cleaning up}
				In the last section we will be looking at how to stop rouge containers and how to get rid of unused containers and images.
			
				\subsubsection{Stop or get killed}
					Containers are meant to be replaceable. 
					If a container becomes inactive, it is easy to replace it with a new one. 
					But how to get rid of a hanging container? 
					Similar to Unix, also Docker has a \textit{kill} command.
					\paragraph{Task}
						\fcolorbox{black}[HTML]{E9F0E9}{\parbox{\linewidth}{%	
							\begin{enumerate}
								\item Download the latest alpine image
								\item Run an alpine container using the -i options, nothing else
								\item In a second terminal, list all running container
								\item Copy the \textit{<container ID}> of the alpine container
								\item Stop or Kill the container
							\end{enumerate}
					}}
					
			\paragraph{Solution}	

				\begin{minipage}{\linewidth}
					\begin{lstlisting}
						docker pull alpine
						docker run -i alpine
						# New terminal
						docker container ls
						#copy <container ID>
						docker stop <container ID> # Alternative: docker kill <container ID>
					\end{lstlisting}
				\end{minipage}

				\subsubsection{Spring cleaning}
					Docker takes a conservative approach to cleaning up unused objects (often referred to as “garbage collection”), such as images, containers, volumes, and networks: these objects are generally not removed unless you explicitly ask Docker to do so. 
					This can cause Docker to use extra disk space. 
					For each type of object, Docker provides a prune command. 
					In addition, you can use docker system prune to clean up multiple types of objects at once. 
					
					\paragraph{Task}
						\fcolorbox{black}[HTML]{E9F0E9}{\parbox{\linewidth}{%
							\begin{enumerate}
								\item Prune images
								\item Prune all images which are not associated to at least one container
								\item Prune container
								\item Prune everything
							\end{enumerate}
						}}
				
					\paragraph{Solution}	
				
						\begin{minipage}{\linewidth}
							\begin{lstlisting}
								docker image prune
								docker image prune -a
								docker container prune
								docker system prune
							\end{lstlisting}
						\end{minipage}
				

\end{document}  %End of document.